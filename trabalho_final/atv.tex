% !TeX spellcheck = pt-BR
\documentclass[
	% -- opções da classe memoir --
	12pt,				% tamanho da fonte
%	openright,			% capítulos começam em pág ímpar (insere página vazia caso preciso)
	oneside,			% para impressão em recto e verso. Oposto a oneside
	a4paper,			% tamanho do papel. 
	% -- opções da classe abntex2 --
	%chapter=TITLE,		% títulos de capítulos convertidos em letras maiúsculas
	%section=TITLE,		% títulos de seções convertidos em letras maiúsculas
	%subsection=TITLE,	% títulos de subseções convertidos em letras maiúsculas
	%subsubsection=TITLE,% títulos de subsubseções convertidos em letras maiúsculas
	% -- opções do pacote babel --
	english,			% idioma adicional para hifenização
	french,				% idioma adicional para hifenização
	spanish,			% idioma adicional para hifenização
	brazil,				% o último idioma é o principal do documento
	%article,
	]{abntex2}


% ---
% PACOTES
% ---

% ---
% Pacotes fundamentais 
% ---
\usepackage{lmodern}			% Usa a fonte Latin Modern
\usepackage[T1]{fontenc}		% Selecao de codigos de fonte.
\usepackage[utf8]{inputenc}		% Codificacao do documento (conversão automática dos acentos)
\usepackage{indentfirst}		% Indenta o primeiro parágrafo de cada seção.
\usepackage{color}				% Controle das cores
\usepackage{graphicx}			% Inclusão de gráficos
\usepackage{microtype} 			% para melhorias de justificação
% ---

% ---
% Pacotes adicionais, usados no anexo do modelo de folha de identificação
% ---
\usepackage{multicol}
\usepackage{multirow}
% ---
	
% ---
% Pacotes adicionais, usados apenas no âmbito do Modelo Canônico do abnteX2
% ---
\usepackage{amsmath}
\usepackage{graphicx}
\graphicspath{ {images/} }
\usepackage[colorinlistoftodos]{todonotes}

% not from template
\usepackage{lipsum}
%\usepackage{tikz}
\usepackage{pgfplots}
\usetikzlibrary{shapes,shapes.geometric,arrows,fit,calc,positioning,automata,patterns}
\usepackage{verbatim}
\usepackage{blindtext}
\usepackage{enumitem}
\usepackage{multicol}
\usepackage{listings}
\newcommand*{\euler}{\mathrm{e}}

\usepackage{fancyvrb}
\usepackage{hyperref}
\usepackage{relsize}

\usepackage{caption}
\usepackage{subcaption}
% ---

% ---
% Pacotes de citações
% ---
\usepackage[brazilian,hyperpageref]{backref}	 % Paginas com as citações na bibl
\usepackage[alf]{abntex2cite}	% Citações padrão ABNT

% --- 
% CONFIGURAÇÕES DE PACOTES
% --- 

% ---
% Configurações do pacote backref
% Usado sem a opção hyperpageref de backref
%\renewcommand{\backrefpagesname}{Citado na(s) página(s):~}
\renewcommand{\backrefpagesname}{}
% Texto padrão antes do número das páginas
\renewcommand{\backref}{}
% Define os textos da citação
\renewcommand*{\backrefalt}[4]{
	\ifcase #1 %
		Nenhuma citação no texto.%
	\or
		Citado na página #2.%
	\else
		Citado #1 vezes nas páginas #2.%
	\fi}%
% ---

% ---
% Informações de dados para CAPA e FOLHA DE ROSTO
% ---
\titulo{Atividade 2}
\autor{Bruno Canale \\ Bruno Giordano \\ Fábio T. Sancinetti \\ Wanderson Ferreira}
\local{São Paulo - Brasil}
\data{07 de dezembro de 2016}
\instituicao{%
  Universidade de São Paulo
  \par
  Escola Politécnica
  \par
  Programa de Pós-Graduação em Engenharia Elétrica \par PPGEE }
\tipotrabalho{Atividade Final}
% O preambulo deve conter o tipo do trabalho, o objetivo, 
% o nome da instituição e a área de concentração 
\preambulo{ \textbf{PSI5886} Princípios de Neurocomputação \break \textbf{Professor}: Emílio Del Moral Hernandez \break \break \textbf{Atividade Final}: Redes Neurais Convolucionais }
% ---

% ---
% Configurações de aparência do PDF final

% alterando o aspecto da cor azul
\definecolor{blue}{RGB}{41,5,195}

% informações do PDF
\makeatletter
\hypersetup{
		%pagebackref=true,
		pdftitle={\@title}, 
		pdfauthor={\@author},
		pdfsubject={\imprimirpreambulo},
		pdfcreator={},
		pdfkeywords={PSI5886}{Redes Neurais}{atv\_final}, 
		colorlinks=true,       		% false: boxed links; true: colored links
		linkcolor=black,          	% color of internal links
		citecolor=black,        	% color of links to bibliography
		filecolor=black,      		% color of file links
		urlcolor=black,
		bookmarksdepth=4
}
\makeatother
% --- 

% --- 
% Espaçamentos entre linhas e parágrafos 
% --- 

% O tamanho do parágrafo é dado por:
\setlength{\parindent}{1.3cm}

% Controle do espaçamento entre um parágrafo e outro:
\setlength{\parskip}{0.2cm}  % tente também \onelineskip

% ---
% compila o indice
% ---
\makeindex
% ---


% --- styles

\tikzstyle{input}=[draw,fill=white!50,circle,minimum size=20pt,inner sep=0pt]
\tikzstyle{hidden}=[draw,fill=white!50,circle,minimum size=20pt,inner sep=0pt]
\tikzstyle{output}=[draw,fill=white!50,circle,minimum size=20pt,inner sep=0pt]
\tikzstyle{bias}=[draw,dashed,fill=gray!50,circle,minimum size=20pt,inner sep=0pt]

\tikzstyle{stateTransition}=[->, thick]


\usepackage{afterpage}

% ----
% Início do documento
% ----
\begin{document}

% Seleciona o idioma do documento (conforme pacotes do babel)
%\selectlanguage{english}
\selectlanguage{brazil}

% Retira espaço extra obsoleto entre as frases.
\frenchspacing 

% ----------------------------------------------------------
% ELEMENTOS PRÉ-TEXTUAIS
% ----------------------------------------------------------
% \pretextual

% ---
% Capa
% ---
%\imprimircapa
% ---

% ---
% Folha de rosto
% (o * indica que haverá a ficha bibliográfica)
% ---
\imprimirfolhaderosto
% ---

% ---
% inserir lista de tabelas
% ---
%\pdfbookmark[0]{\listtablename}{lot}
%\listoftables*
%\cleardoublepage
% ---

% ---
% inserir lista de abreviaturas e siglas
% ---
\begin{siglas}
	\item[CNN] \textit{Convolutional Neural Networks} - Redes Neurais Convolucionais
	\item[GPU] \textit{Graphic Processor Unit} - Unidade de Processamento Gráfico
	\item[MLP] \textit{Multilayer Perceptron} - Perceptron em muti-camadas
	\item[RNA] Redes Neurais Artificiais
	\item[RNN] \textit{Recurrent Neural Networks} - Redes Neurais Recorrentes
\end{siglas}
% ---

% ---
% inserir lista de símbolos
% ---
%\begin{simbolos}
%  \item[$ \Gamma $] Letra grega Gama
%  \item[$ \Lambda $] Lambda
%  \item[$ \zeta $] Letra grega minúscula zeta
%  \item[$ \in $] Pertence
%\end{simbolos}
% ---

% ---
% inserir o sumario
% ---
\pdfbookmark[0]{\contentsname}{toc}
\tableofcontents*
%\cleardoublepage

% ---


% ----------------------------------------------------------
% ELEMENTOS TEXTUAIS
% ----------------------------------------------------------
\textual

% ----------------------------------------------------------
% Introdução (exemplo de capítulo sem numeração, mas presente no Sumário)
% ----------------------------------------------------------
\chapter[Introdução]{Introdução}

\par \lipsum[50]

\chapter{LIPSUM}
\section{Wolverinis Adamantium}
\par \lipsum[30] \cite{deeplearning_net_mlp}

\par Exemplo de citação em linha segundo \citeonline{rojas1996} aqui.



% ---
% Conclusão
% ---
\chapter{Conclusão}
% ---

\par \lipsum[30-30]


% ----------------------------------------------------------
% ELEMENTOS PÓS-TEXTUAIS
% ----------------------------------------------------------
%\postextual

% ----------------------------------------------------------
% Referências bibliográficas
% ----------------------------------------------------------
\bibliography{bibliografia}

% ----------------------------------------------------------
% Glossário
% ----------------------------------------------------------
%
% Consulte o manual da classe abntex2 para orientações sobre o glossário.
%
%\glossary

% ----------------------------------------------------------
% Apêndices
% ----------------------------------------------------------

% ---
% Inicia os apêndices
% ---
%\begin{apendicesenv}

% Imprime uma página indicando o início dos apêndices
%\partapendices

% ----------------------------------------------------------
%\chapter{Quisque libero justo}
% ----------------------------------------------------------

%\lipsum[50]

% ----------------------------------------------------------
%\chapter{Nullam elementum urna vel imperdiet sodales elit ipsum pharetra ligula
%ac pretium ante justo a nulla curabitur tristique arcu eu metus}
%% ----------------------------------------------------------
%\lipsum[55-57]
%
%\end{apendicesenv}
%% ---
%
%
%% ----------------------------------------------------------
%% Anexos
%% ----------------------------------------------------------
%
%% ---
%% Inicia os anexos
%% ---
%\begin{anexosenv}
%
%% Imprime uma página indicando o início dos anexos
%\partanexos
%
%% ---
%\chapter{Morbi ultrices rutrum lorem.}
%% ---
%\lipsum[30]
%
%% ---
%\chapter{Cras non urna sed feugiat cum sociis natoque penatibus et magnis dis
%parturient montes nascetur ridiculus mus}
%% ---
%
%\lipsum[31]
%
%% ---
%\chapter{Fusce facilisis lacinia dui}
%% ---
%
%\lipsum[32]
%
%\end{anexosenv}
%
%%---------------------------------------------------------------------
%% INDICE REMISSIVO
%%---------------------------------------------------------------------
%
%\phantompart
%
%\printindex
%
%%---------------------------------------------------------------------
%% Formulário de Identificação (opcional)
%%---------------------------------------------------------------------
%\chapter*[Formulário de Identificação]{Formulário de Identificação}
%\addcontentsline{toc}{chapter}{Exemplo de Formulário de Identificação}
%\label{formulado-identificacao}
%
%Exemplo de Formulário de Identificação, compatível com o Anexo A (informativo)
%da ABNT NBR 10719:2015. Este formulário não é um anexo. Conforme definido na
%norma, ele é o último elemento pós-textual e opcional do relatório.
%
%\bigskip
%
%\begin{tabular}{|p{9cm}|p{5cm}|}
%\hline
%\multicolumn{2}{|c|}{\textbf{\large Dados do Relatório Técnico e/ou científico}}\\
%\hline
%\multirow{4}{10cm}[24pt]{Título e subtítulo}& Classificação de segurança\\
%                   & \\
%                   \cline{2-2}
%                   & No.\\
%                   & \\
%				
%\hline
%Tipo de relatório & Data\\
%\hline
%Título do projeto/programa/plano & No.\\
%\hline
%\multicolumn{2}{|l|}{Autor(es)} \\
%\hline
%\multicolumn{2}{|l|}{Instituição executora e endereço completo} \\
%\hline
%\multicolumn{2}{|l|}{Instituição patrocinadora e endereço completo} \\
%\hline
%\multicolumn{2}{|l|}{Resumo}\\[3cm]
%\hline
%\multicolumn{2}{|l|}{Palavras-chave/descritores}\\
%\hline
%\multicolumn{2}{|l|}{
%Edição \hfill No. de páginas \hfill No. do volume \hfill Nº de classificação \phantom{XXXX}} \\
%\hline
%\multicolumn{2}{|l|}{
%ISSN \hfill \hfill Tiragem \hfill Preço \phantom{XXXXXXXX}} \\
%\hline
%\multicolumn{2}{|l|}{Distribuidor} \\
%\hline
%\multicolumn{2}{|l|}{Observações/notas}\\[3cm]
%\hline
%\end{tabular}

\end{document}
