\documentclass[11pt]{beamer}
\usepackage[utf8]{inputenc}
\usepackage[T1]{fontenc}
\usetheme{default}
\usepackage{lipsum}
\begin{document}
	%\author{}
	%\title{}
	%\subtitle{}
	%\logo{}
	%\institute{}
	%\date{}
	%\subject{}
	%\setbeamercovered{transparent}
	%\setbeamertemplate{navigation symbols}{}
	\frame[plain]{\maketitle}
	
	\begin{frame}
		\frametitle{Redes Neurais Convolucionais}
		\begin{itemize}
			\item Utilização:
				\begin{itemize}
					\item teste
				\end{itemize}
		\end{itemize}
			
	\end{frame}


	\begin{frame}
		\frametitle{Redes Deconvolucionais}
		Redes Deconvolucionais buscam gerar o sinal de entrada pela soma 
Deconvolutional Network
is top-down; it seeks to generate the input signal by a sum
over convolutions of the feature maps (as opposed to the
input) with learned filters.		
	\end{frame}
	\begin{frame}
	\frametitle{Redes Deconvolucionais}
	\begin{figure}
		\centering
		\includegraphics[width=0.9\linewidth]{../images/conv_deconv_noh_learning}
		\caption{}
		\label{fig:convdeconvnohlearning}
	\end{figure}
		
\end{frame}

\end{document}