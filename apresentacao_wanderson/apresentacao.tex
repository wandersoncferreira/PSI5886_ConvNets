\documentclass[11pt]{beamer}
\usepackage[utf8]{inputenc}
\usepackage[T1]{fontenc}
\usetheme{default}
\usepackage{lipsum}
\begin{document}
	%\author{}
	%\title{}
	%\subtitle{}
	%\logo{}
	%\institute{}
	%\date{}
	%\subject{}
	%\setbeamercovered{transparent}
	%\setbeamertemplate{navigation symbols}{}
	\frame[plain]{\maketitle}

    \begin{frame}
      \frametitle{Objetivos do Trabalho}
    \end{frame}

    \begin{frame}
      \frametitle{Introdução e Motivação}
    \end{frame}

    \begin{frame}
      \frametitle{Apresentação do Bruno}
    \end{frame}

    \begin{frame}
      \frametitle{Apresentação do Bruno Canale}
    \end{frame}

	\begin{frame}
      \frametitle{Python - Keras Framework para Machine Learning}
      Keras was initially developed as part of the research effort of
      project ONEIROS (Open-ended Neuro-Electronic Intelligent Robot
      Operating System)

      Example of MLP:
      modelo = Sequential()
      modelo.add(Dense(numero_neuronios, input_dim=treino.shape[1],
      init='uniform', activation='tanh'))
      modelo.add(Dense(numero_neuronios=1, activation='linear'))
      modelo.compile(learning_rate=0.01, momentum=0.9,
      optimizer=stochastic_gradient_descend)
      modelo.fit(features_treino, target_treino, numero_epocas=40)
      modelo.evaluate(features_teste, target_teste)
      
    \end{frame}

    \begin{frame}
      \frametitle{Base de dados utilizada}

      Descrição sobre o MINST
    \end{frame}

    \begin{frame}
      \frametitle{Implementação da Rede Convolucional}
      como foi implementada a rede convolucional
    \end{frame}

    \begin{frame}
      \frametitle{Apresentacao do Fábio - Sugestão: Introdução a exploração dos resultados}
    \end{frame}

    \begin{frame}
      \frametitle{Apresentacao do Fábio - Sugestao: Apresentacao das
        imagens dos filtros e saidas das convolucoes}
    \end{frame}

    \begin{frame}
      \frametitle{Apresentacao do Fábio - Sugestão: Discussao sobre
        como essas observacoes ligam na MLP clássica e/ou problemas de
        Machine Learning}
    \end{frame}
    
\end{document}